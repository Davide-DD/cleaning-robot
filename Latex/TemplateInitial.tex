\documentclass{llncs}
%%%%%%%%%%%%%%%%%%%%%%%%%%%%%%%%%%%%%%%%%%%%%%%%%%%%%%%%%%%
%% package sillabazione italiana e uso lettere accentate
\usepackage[italian]{babel}
\usepackage[T1]{fontenc}
\usepackage[utf8]{inputenc}
%%%%%%%%%%%%%%%%%%%%%%%%%%%%%%%%%%%%%%%%%%%%%%%%%%%%%%%%%%%%%

\usepackage{url}
\usepackage{xspace}
\usepackage{color}
\makeatletter
%%%%%%%%%%%%%%%%%%%%%%%%%%%%%% User specified LaTeX commands.
\usepackage{manifest}

\makeatother


%%%%%%%
 \newif\ifpdf
 \ifx\pdfoutput\undefined
 \pdffalse % we are not running PDFLaTeX
 \else
 \pdfoutput=1 % we are running PDFLaTeX
 \pdftrue
 \fi
%%%%%%%
 \ifpdf
 \usepackage[pdftex]{graphicx}
 \else
 \usepackage{graphicx}
 \fi
%%%%%%%%%%%%%%%
 \ifpdf
 \DeclareGraphicsExtensions{.pdf, .jpg, .tif}
 \else
 \DeclareGraphicsExtensions{.eps, .jpg}
 \fi
%%%%%%%%%%%%%%%

\newcommand{\java}{\textsf{Java}}
\newcommand{\android}{\texttt{Android}}
\newcommand{\dsl}{\texttt{DSL}}
\newcommand{\jazz}{\texttt{Jazz}}
\newcommand{\rtc}{\texttt{RTC}}
\newcommand{\ide}{\texttt{Contact-ide}}
\newcommand{\xtext}{\texttt{XText}}
\newcommand{\xpand}{\texttt{Xpand}}
\newcommand{\xtend}{\texttt{Xtend}}
\newcommand{\pojo}{\texttt{POJO}}
\newcommand{\junit}{\texttt{JUnit}}

\newcommand{\action}[1]{\texttt{#1}\xspace}
\newcommand{\codescript}[1]{{\scriptsize{\texttt{#1}}}\xspace}
\newcommand{\code}[1]{{\color{blue}\small{\texttt{#1}}}}
\newcommand{\fname}[1]{\small{\color{magenta}\texttt{#1}}}
\newcommand{\node}{\textsf{NodeJs}}
\newcommand{\qa}{\textsf{\textit{QActor}}}

% Cross-referencing
\newcommand{\labelsec}[1]{\label{sec:#1}}
\newcommand{\xs}[1]{\sectionname~\ref{sec:#1}}
\newcommand{\xsp}[1]{\sectionname~\ref{sec:#1} \onpagename~\pageref{sec:#1}}
\newcommand{\labelssec}[1]{\label{ssec:#1}}
\newcommand{\xss}[1]{\subsectionname~\ref{ssec:#1}}
\newcommand{\xssp}[1]{\subsectionname~\ref{ssec:#1} \onpagename~\pageref{ssec:#1}}
\newcommand{\labelsssec}[1]{\label{sssec:#1}}
\newcommand{\xsss}[1]{\subsectionname~\ref{sssec:#1}}
\newcommand{\xsssp}[1]{\subsectionname~\ref{sssec:#1} \onpagename~\pageref{sssec:#1}}
\newcommand{\labelfig}[1]{\label{fig:#1}}
\newcommand{\xf}[1]{\figurename~\ref{fig:#1}}
\newcommand{\xfp}[1]{\figurename~\ref{fig:#1} \onpagename~\pageref{fig:#1}}
\newcommand{\labeltab}[1]{\label{tab:#1}}
\newcommand{\xt}[1]{\tablename~\ref{tab:#1}}
\newcommand{\xtp}[1]{\tablename~\ref{tab:#1} \onpagename~\pageref{tab:#1}}
% Category Names
\newcommand{\sectionname}{Section}
\newcommand{\subsectionname}{Subsection}
\newcommand{\sectionsname}{Sections}
\newcommand{\subsectionsname}{Subsections}
\newcommand{\secname}{\sectionname}
\newcommand{\ssecname}{\subsectionname}
\newcommand{\secsname}{\sectionsname}
\newcommand{\ssecsname}{\subsectionsname}
\newcommand{\onpagename}{on page}

\newcommand{\xauthA}{Andrea Boscarino}
\newcommand{\xauthB}{Davide Di Donato}
\newcommand{\xauthC}{Federico Livi}
\newcommand{\xfaculty}{II Faculty of Engineering}
\newcommand{\xunibo}{Alma Mater Studiorum -- University of Bologna}
\newcommand{\xaddrBO}{viale Risorgimento 2}
\newcommand{\xaddrCE}{via Venezia 52}
\newcommand{\xcityBO}{40136 Bologna, Italy}
\newcommand{\xcityCE}{47023 Cesena, Italy}

%
% Comments
%
\newcommand{\todo}[1]{\bf{TODO:}\emph{#1}}


\begin{document}

\title{Software Engineering\\ Final Case Study 2017-2018 }

\author{\xauthA, \xauthB, \xauthC }

\institute{%
  \xunibo\\\xaddrBO, \xcityBO\\\email{andrea.boscarino@studio.unibo.it\\ davide.didonato3@studio.unibo.it\\ federico.livi2@studio.unibo.it }
}

\maketitle
\sloppy

%===========================================================================
\section{Introduction}
\labelsec{intro}
Questa relazione descrive lo sviluppo di un sistema software proposto dal case study, ponendo particolare attenzione all'intero processo di produzione. \\
La descrizione del processo di sviluppo seguirà un approccio top down. \\
Lo scopo di questo progetto è realizzare un sistema software distribuito per controllare un robot usato per pulire il pavimento di una stanza.
In particolare il robot è un robot a trazione differenziale (DDR) e deve essere controllabile da un utente autorizzato attraverso un'interfaccia remota. \\ Il DDR parte da una posizione iniziale ed arriva ad una posizione finale (rilevate da due sonar differenti) schivando ostacoli fissi e mobili lungo il cammino. Il robot lavora solo sotto determinate condizioni e mentre si muove un led (fisico o virtuale) lampeggia per segnalare il suo stato di attività. 
%===========================================================================
 
%===========================================================================
\section{Vision}
Dalle tecnologie alla analisi e al progetto logico e ritorno alle tecnologie.
\labelsec{vision}
%=========================================================================== 
 
%===========================================================================
\section{Requirements}
The problem now is the following:
with reference to a mbot physical robot working in virtual environment, build an application that sends to the
radar the data sensed by the virtual and the real sonars. More specifically:
\begin{itemize}
\item the data of the virtual sonar sonar1 must be displayed on the direction of angle=30;
\item the data of the virtual sonar sonar2 must be displayed on the direction of angle=120;
\item the data of the virtual sonar on the virtual robot must be displayed on the direction of angle=90 at the fixed
distance of 40;
\item the data of the real sonar on the physical robot must be displayed on the direction of angle=0;
\end{itemize}
\labelsec{Requirements}
%===========================================================================

%===========================================================================
\section{Requirement analysis}
\labelsec{ReqAnalysis}
%===========================================================================
 

%===========================================================================
\section{Problem analysis}
\labelsec{ProblemAnalysis}
%===========================================================================


%===========================================================================
\section{Project}
\labelsec{Project}
%===========================================================================


%===========================================================================
\section{Implementation}
\labelsec{Implementation}
%===========================================================================

%===========================================================================
\section{Testing}
\labelsec{Testing}
%===========================================================================

%===========================================================================
\section{Maintenance}
\labelsec{Maintenance}
%===========================================================================

%===========================================================================
\section{Deployment}
\labelsec{Deployment}
%===========================================================================
 
%===========================================================================
\section{Author}
\labelsec{Author}
%===========================================================================

\vskip.5cm
%%% \begin{figure}
\begin{tabular}{ | c |  }
\hline
  % after \\: \hline or \cline{col1-col2} \cline{col3-col4} ...
  Photo of the author 
  \\
\hline
   \includegraphics[scale = 0.7]{img/foto_autore.jpg}
  \\
\hline
\end{tabular}
 
\end{document}












